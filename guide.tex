% 名城大学理工学部情報工学科卒業研究発表会
% ・名城大学理工学研究科情報工学専攻公聴会
% アブストラクトサンプル
% Last Update: 2021/11/13

\documentclass[a4paper, 9pt]{jarticle}
\usepackage{ieabst}
\usepackage{newenum}
\usepackage[font=small]{caption}
\usepackage[dvipdfmx]{graphicx}

%タイトルが長い場合は,勝手に改行されます.
%% 「\\」の挿入で任意の位置に改行を入れられます.
\題目{ヒント数17の数独パズルの効率的な生成に関する研究}
\学籍番号{223426015}
\氏名{長尾 卓}
\研究室{山本}
% 旭研究室 宇佐見研究室 亀谷研究室 川澄研究室 小中研究室
% 佐川研究室 鈴木研究室 高比良研究室 田中研究室 寺本研究室
% 中野研究室 野崎研究室 坂野研究室 水沼研究室 向井研究室
% 柳田研究室 山田(啓)研究室 山田(宗)研究室 山本研究室
% 吉川研究室 米澤研究室


%1ページに入りきらない場合は,下の数字を少し小さめに変更.
\renewcommand{\baselinestretch}{1}

\begin{document}
\small

\twocolumn[\vspace*{29mm}] %タイトルが2行の場合は29mm→36mmに.
\begin{論文概要}           %この行は消してはいけません

\section{はじめに}
数独パズルは,ペンシルパズルの一種である.
ペンシルパズルとは,問題に対して答えを徐々に鉛筆で書き込んでいき,
答えを導くようなパズルのことである.
ペンシルパズルには,数独パズルのほかにスリザーリンクや虫食い算などが
知られている.数独パズルは,与えられたヒント
(例:\figurename{\ref{fig:problem_and_answer}}左)から,
1から9の数字を用いて縦,横,$3 \times 3$ブロックの
どの数字にも重複させないように,マスを埋めていくパズルである.
\begin{figure}[b]
  \begin{minipage}[b]{0.49\linewidth}
    \centering
    \includegraphics[width=3.5cm]{prob.png}
  \end{minipage}
  \begin{minipage}[b]{0.49\linewidth}
    \centering
    \includegraphics[width=3.5cm]{ans.png}
  \end{minipage}
  \caption{ヒント数17の数独パズル(左)とその解(右).}
  \label{fig:problem_and_answer}
\end{figure}
\figurename{\ref{fig:problem_and_answer}}左で与えられる問題の
答えは\figurename{\ref{fig:problem_and_answer}}右である.
また,1つの問題から得られる最終盤面はただ1通りである必要がある.
以降,数独パズルの最終盤面を「解」と表す.
本研究には先行研究 \cite{previous_research} がある.
先行研究の目的は,ヒント数が少ない
数独パズルの問題を確率的に効率よく生成することである.
先行研究が提案するヒント生成アルゴリズムの
ヒント数17の問題生成割合は約5\%であり,
1つの問題を生成する平均時間
\footnote{プログラムの実行環境は以下の通りである: 
OS:Linux Ubuntu 16.04.7,
CPU:Intel(R) Xeon(R) CPU E5-2640 v4 @ 2.40GHz,
メモリ:64G,
コンパイラ:gcc5.4.0.
}
は約105分であった.
なお,数独パズルにおけるヒント数の下限は17個であることが 
\cite{seventeen_hints} により証明されている.
本研究ではヒント数17の問題を効率よく生成することを目的とする.
本研究が提案するヒント生成アルゴリズムは,先行研究の
ヒント生成アルゴリズムを改善したものである.
そのヒント数17の問題生成割合は約93\%であり,
先行研究と同環境下で1つの問題を生成する平均時間は約11分であった.
また,生成したヒント数17の問題の中に異なる問題がどの程度
含まれているか調べた結果,200問中$166 \sim 185$問が含まれていた.

\section{本研究で用いる手法}
本研究ではシミュレイテッド・アニーリング(SA) \cite{sa} と,
ビームサーチ(BS) \cite{beam_search},Algorithm X(AX) \cite{AX} の
3つの手法を用いる.以下にそれらの説明をする.

% SAは目的の定常分布をもつマルコフ連鎖を
% 安定して構成するヒューリスティックなアルゴリズムである.
% マルコフ連鎖上のある状態からランダムな近傍状態に遷移させる
% ことを繰り返し,マルコフ連鎖を定常状態にした後に,
% SAのパラメータである内部温度$T$を微減させて
% 定常分布を目的の定常分布へ緩く変化させる.
% 以上を繰り返すことで,目的の定常分布をもつマルコフ連鎖を構成していく.
% そして,目的の定常分布を得るまでに,
% 得たい状態をサンプリングすることが可能である.
% 一般的なSAは,ある状態の出現確率をボルツマン因子により決定し,
% 近傍状態への遷移にはメトロポリス法 \cite{sa} を用いる.
% 本研究もボルツマン因子やメトロポリス法を用いる.
% BSは,木構造上の根からスタートする$w$本のパスを考えて,
% それらのパスを同時に葉の方向に伸ばしながら評価値がより良いノードを
% 探索するヒューリスティックな探索アルゴリズムである.
% BSの性質は$w$によって大きく変化する.
% $w$が大きい場合は,空間的計算量と時間的計算量の両方が大きくなる.
% 一方で,$w$が小さい場合は,最適解や最適解に近いノードへのパスが
% 途中で絶たれる可能性が大きくなる.
SAは,ある状態からランダムな近傍状態に遷移させながら,
最適解を得るアルゴリズムである.内部温度というパラメータを設定し,
徐々に内部温度を低下させていき,内部温度に依存した遷移確率で
状態の遷移を行う.本研究では,メトロポリス法を用いて
マルコフ連鎖を構成し,マルコフ連鎖上でSAを行う.
BSは,木構造上の根からスタートする$w$本のパスを考えて,
それらのパスを同時に葉の方向に伸ばしながら評価値がより良いノードを
探索するヒューリスティックな探索アルゴリズムである.
BSの性質は$w$によって大きく変化するため,慎重に設定する必要がある.
AXは,集合$S$と複数の部分集合$s$をもつ厳密被覆問題をバックトラックで
効率よく解くアルゴリズムである.ある時点までに選択してきた全ての$s$と
互いに素の$s$のデータのみを保持し,保持している複数の$s$から適切に$s$を選択
することを再帰的に繰り返す.保持している複数の$s$のみでは
$S$の要素をカバーできないと判断した場合に,効率よく枝狩りを行うため効率がよい.

\section{先行研究のヒント生成アルゴリズム}
本研究が提案するヒント生成アルゴリズムは \cite{previous_research} の方法を
改善したものである.先行研究のヒント生成アルゴリズムは,
まず,ヒント集合$H^{(0)} = \emptyset$を用意する.
$H^{(i)}$の右肩の数字が表記されている場合は,
$|H| = i$を表す.
次に,適切なヒント$h$を$H$に添加していき,
$H$から得られる解の集合$S(H)$の大きさ$|S(H)|$を徐々に減少させていく.
$h$はマスの位置と数字の組である.
適切な$h$とは,$H^{(i)}$に添加して得られる$H^{(i+1)}$の
$|S(H^{(i+1)})|$が最小となる$h$のことである.
最終的には,$|S(H)| = 1$となるまで$h$を$H$に添加して問題を生成する.
$h$の選択方法は,$H$から$S(H)$を得て,解に出現する場所と
数字の組である複数の要素のうち最も出現回数が少ない要素を$h$とする.
なお,$H$から$S(H)$を得るためには,
バックトラック(BT)を用いる.
一方で,$H^{(i)} ~ (i < 14)$は十分に$|S(H^{(i)})|$が減少しておらず,
BTで解を列挙することは効率が悪い.
そのため,SAを用いることで$H^{(i)}$から得られる解を偏りなく
等確率に多く生成することを行い,確率的に$h$を選択する.

\section{先行研究のヒント生成アルゴリズムに加えた改善}
先行研究のヒント生成アルゴリズムに主に4点の改善を加えた.
2点はヒント数17の問題生成割合を高めるための
改善で,残りの2点はヒント生成の高速化のために行った改善である.
ヒント数17の問題生成割合を高めるための改善の1点目は,
生成した$H^{(14)}$に3つの$h$をまとめて添加するようにした点である.
これにより,生成してきた$H^{(14)}$が,あるヒント数17の問題の部分集合である
場合は,必ずそのヒント数17の問題を生成できるようになる.
この改善のみを加えた場合のヒント数別の問題生成割合は
約5\%向上して約10\%になった.
2点目は,$H^{(i)}$の解集合の大きさを柔軟に減少させていくために,
解集合が最小の$w$個の$H^{(i)}$を
保持していくBSを用いた点である \cite{nagao}.
生成した$H^{(14)}$の解集合が小さいときほど,
$H^{(14)}$に3つのヒントを添加する場合に,
ヒント数17の問題が生成しやすくなることは実験により分かっている.
本研究で行うBSには一般的なBSにはないパラメータがあり,
1つの$H^{(i)}$から構築する$H^{(i+1)}$の個数の上限$\rho$を設けている.
パラメータ$\rho$は,解集合が減少していきづらい$H^{(i)}$から構築された
$H^{(i+1)}$のみをBSで保持することを避ける目的がある.
\begin{figure}[bt]
  \centering
  \includegraphics[keepaspectratio, scale=0.4]{best_beam_width_and_rho.png}
  \caption{ビーム幅$w$と上限$\rho$~$(2 \leq \rho \leq w \leq 10)$により,
  生成した$w$個の$H^{(14)}$の中で最小の解集合の大きさが
  $160,000$以下になる確率を円の大きさで表した図.
  生成割合が最も高い$(w, \rho)$の組は赤色で表す.} 
  \label{fig:best_beam_width_and_rho}
\end{figure}
\figurename{\ref{fig:best_beam_width_and_rho}}より,
最も柔軟に解集合が減少するパラメータ$(w, \rho)$は
実験した範囲$(2 \leq \rho \leq w \leq 10)$では,
$(w, \rho) = (10, 2)$であった.
% 実験した$w$と$\rho$~$(2 \leq \rho \leq w \leq 10)$の範囲では,
% 最も柔軟に解集合が減少するパラメータは$(w, \rho) = (10, 2)$であった.
$w$を大きくすると,さらに柔軟に解集合は減少しやすくなると予想する.
ヒント生成の高速化のための改善の1点目は,
数独パズルが厳密被覆問題と見なすことができることから,
$H^{(i)} ~ (i \geq 14)$から解集合を得るためにBTではなく,
より効率の良いAXを用いるようにした点である.
この変更により,$H^{(14)}$にまとめて添加する3つのヒントの選択時間は
約$66 \sim 80$\%ほど短縮した.
2点目は,$H$から解をサンプリングするSAにおける
近傍状態を変更した点である.
先行研究のSAは,ランダムに選択した行の中で,
ヒント以外の2マスの数字を交換してできる盤面を近傍状態としていた.
一方で,本研究のSAは,ランダムに選択した$3 \times 3$ブロックの中で,
ヒント以外の2マスの数字を交換してできる盤面を近傍状態とする.
ただし,初めに$H$を考慮して,先行研究は全行に,本研究は全ブロックに
1から9の数字が無作為に書き込まれた初期盤面から遷移を行うようにする.
この変更により,SAの解のサンプリング効率は約5倍向上した.

\section{改善によるヒント数17の問題生成の割合と効率の変化}
前節で述べたような,高速化1,2を用いることを前提とし,ヒント数17の問題生成割合を高めるための改善の2点
の組み合わせで,どの程度ヒント数17の問題生成の割合と効率が変化するか調べた.
ヒント数17の問題生成割合を\figurename{\ref{fig:w_rho_prob_17hint}}に示し,
1つのヒント数17の問題を生成する平均秒数を
\figurename{\ref{fig:w_rho_efficiency_17hint}}に示す.
% ヒント数17の問題生成割合の結果を\figurename{\ref{fig:}}に示し,
% 1つのヒント数17の問題を生成する平均時間を\figurename{\ref{fig:}}に示す. %
\begin{figure}[bt]
  \centering
  \includegraphics[keepaspectratio, scale=0.3]{w_rho_prob_17hint.png}
  \caption{$w,\rho$~$(2 \leq w \leq 10,\rho:2,w)$と
  $H^{(14)}$にヒントを3つ(three)または1つずつ(single)ヒントを
  添加する場合のヒント数17の問題生成割合.} 
  \label{fig:w_rho_prob_17hint}
\end{figure}
\begin{figure}[bt]
  \centering
  \includegraphics[keepaspectratio, scale=0.4]{w_rho_efficiency_17hint.png}
  \caption{\figurename{\ref{fig:w_rho_prob_17hint}}の結果より
  1つのヒント数17の問題を生成する平均秒数を求めて示した図.} 
  \label{fig:w_rho_efficiency_17hint}
\end{figure}
\figurename{\ref{fig:w_rho_prob_17hint}}より,
$w$が大きいほどかつ,$H^{(14)}$に1つずつではなく
3つのヒントをまとめて添加する場合が,ヒント数17の問題生成割合は
高くなることが分かった.
一方で,\figurename{\ref{fig:w_rho_efficiency_17hint}}からは
$H^{(14)}$以降のヒント生成方法の差異によっては
ヒント数17の問題生成効率に目立った違いがないことが分かった.
つまり,$H^{(14)}$に1つずつではなく
3つのヒントをまとめて添加する方が計算量が大きくなることが分かる.
\figurename{\ref{fig:w_rho_efficiency_17hint}}の中で
最もヒント数17の問題生成効率が良かった場合は,
$(w, \rho) = (10, 2)$かつ$H^{(14)}$に3つのヒントをまとめて添加する
場合であり,1つのヒント数17の問題を生成する平均時間は736秒であった.
また,生成したヒント数17の問題の中に異なる問題がどの程度
含まれているか調べた結果,200問中$166 \sim 185$問の異なる問題が含まれていた.

\section{まとめと今後の課題}
先行研究のヒント生成アルゴリズムに
ビームサーチやAlgorithm Xを用いることと,
$H^{(14)}$に3つのヒントをまとめて添加する変更を行い,
ヒント数17の問題生成効率を改善した.
問題生成に要する平均時間とヒント数17の問題生成割合は
先行研究が約105分と約5\%であったのに対し,
本研究では約11分と約93\%であった.
この本研究の結果は$(w, \rho) = (10, 2)$かつ$H^{(14)}$に
3つのヒントをまとめて添加する場合のものである.
また,生成した多くのヒント数17の問題の中に異なる問題がどの程度
含まれているか調べ,200問中$166 \sim 185$問が含まれていた.
今後の課題は,$w$や$\rho$をより広い範囲で実験して,
より適切な$(w, \rho)$の組を見つけることや,
より等確率にヒント数17の問題を生成するように
アルゴリズムを改善することである.

\begin{thebibliography}{10}

  \bibitem{previous_research}
  古川 湧:ヒントの少ない数独パズルの生成に関する研究.$2020$年度名城大学大学院理工学研究科修士論文(2021).

  \bibitem{seventeen_hints}
  G. McGuire, B. Tugemann, and G. Civario:
  There is no 16-clue Sudoku: Solving the Sudoku minimum number of clues problem via hitting set enumeration.
  {\it Experimental Mathematics}, 23:2, pp. 190--217 (2014).

  \bibitem{sa}
  Bruce E. Rosen,中野 良平:シミュレーテッドアニーリング:ー基礎と最新技術ー,
  人工知能学会誌,Vol.9,No.3,pp.365--372(1994).
  
  \bibitem{beam_search}
  Y. Inoue and S. Minato: Acceleration of ZDD Construction for Subgraph 
  Enumeration via Path-width Optimization. TCS Technical Report, 
  TSC-TR-A-16-80, (2016).

  \bibitem{AX}
  D. E. Knuth: Dancing links. 
  {\it Millennial Perspectives in Computer Science}, pp. 187--214 (2000).
  
  \bibitem{nagao}
  長尾 卓,山本 修身:ビームサーチを用いたヒント数17の数独パズルの効率的な生成について.
  {\it ゲームプログラミングワークショップ2022論文集},pp. 96--103 (2022).

\end{thebibliography}

\end{論文概要} %この行は消してはいけません
\end{document}
